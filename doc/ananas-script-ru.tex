\documentclass[11pt,russian,a4paper]{article}
\usepackage[utf8]{inputenc}
\usepackage[T1]{fontenc}
\usepackage[russian]{babel}
\usepackage{graphics}
%\oddsidemargin=-2cm
\pagestyle{empty}
\author{}
\title{В.А.Гражданкин\\
\includegraphics{ananas-text.jpg}\\
``Ананас-Скрипт''\\
Руководство по встроенному языку}
\makeindex
%\makeatletter

\begin{document}
\maketitle

%\newenvironment{code}{\begin{verbatim}}{\end{verbatim}}
\newcommand{\AS}{Ananas Script }
\newcommand{\A}{Ananas}
%\newcommand{\AS}{Ananas script for applications}
\newcommand{\Q}{script}
\newcommand{\QS}{script}
%\newcommand{\BR}{\\}
%\newcommand{%\omit}{\\}
%\newcommand{%\endomit}{\\}
%\newcommand{\i}[1]{\index{#1}}
%\newcommand{\l}[1]{\index{#1}}
%\newcommand{\code}{\begin{verbatim}}
%\newcommand{\code}{\end{verbatim}}
%\newcommand{\e}{\sf}
%\renewcommand{\c}{\sf}
%\renewcommand{\bf}{\si}
%\newcommand{\section1}[1]{\section{#1}}
%\newcommand{\section2}[1]{\subsection{#1}}
%\newcommand{\section3}[1]{\subsubsection{#1}}

\newpage
\abstract

Данное руководство предназначено для прикладных программистов, настройщиков, осуществляющих разработку, внедрение
прикладных решений на основе платформы Ананас. Осуществляющих сопровождение систем автоматизации учетной деятельности.
\vfill{}
(с) 2004г.
\newpage
\tableofcontents
\newpage

\section{Введение}


\section{Основные идеи}


\section{Основы синтаксиса}

\subsection{Комментарии}

\AS поддерживает такой же синтаксис комментариев, как и C++. Однострочные
комментарии располагаются в начале или после операторов на той же строке.
Многострочные комментарии могут располагаться где угодно.

\begin{verbatim}
// Однострочный комментарий.

/*
    Многострочный
    комментарий.
*/
\end{verbatim}


\subsection{Идентификаторы}

Идентификаторы совпадают с регулярным выражением \verb! [_A-Za-z][_A-Za-z0-9]!.
Идентификаторы используются для переменных, констант, имен классов, имен функций
и меток.

Существуют зарезервированные слова, которые нельзя использовать в качестве
идентификаторов. Смотри полный список в главе {Встроенные функции и операторы}.

\subsection{Объявление классов, функций, констант, переменных}


\subsubsection{Класс}

\AS полностью объектно ориентированный язык. Классы могут быть объявлены
в исходных модулях как показано ниже.

Пример:

\begin{verbatim}
    class Circle {
        var x;
        var y;
        var r;

        function Circle( posx, posy, radius )
        {
            this.x = posx;
            this.y = posy;
            this.r = radius;
        }
        function setX( posx ) { this.x = posy; }
        function setY( posy ) { this.y = posy; }
        function setR( radius ) { this.r = radius; }
        function x() { return this.x; }
        function y() { return this.y; }
        function r() { return this.r; }
    }

    class ColorCircle extends Circle {
        var rgb;

        function ColorCircle( posx, posy, radius, rgbcolor)
        {
            Circle( posx, posy, radius );
            this.rgb = rgbcolor;
        }
        function setRgb( rgbcolor ) { rgb = rgbcolor; }
        function rgb() { return rgb; }
    }

\end{verbatim}

Конструктор класса является функцией, имя которой совпадает с именем класса
(Регистро-чувствительная). Конструктор не должен содержать
выражения return; он автоматически возвращает объект соответствующего
типа.

\AS не имеет функции деструктора ( функции, которая вызывается когда
класс уничтожается) для класса.

Переменные-члены класса объявляются через var, члены-функции (методы) - через
function.

Экземпляр объекта ссылается на себя с использованием оператора this.

Внутри методов класса члены-переменные и члены-функции могут быть
явно доступны через this (например this.x = posx;).
Это не требуется, но иногда повышает наглядность.

\AS поддерживает одиночное наследование,и если класс наследуется от другого класса
конструктор родителя может быть вызван функцией super().

Функции и переменные могут определять типы их аргументов
и тип возвращаемого результата если необходимо.

Например:
\begin{verbatim}
    class ColorCircle extends Circle {
        var _rgb : Color;

        function ColorCircle(
                x : Integer, y : Integer, r : Integer,
                rgb : Color )
        {
                super( x, y, r );
                this._rgb = rgb;
        }
        function setRgb( rgb : Color ) { this._rgb = rgb; }
        function rgb() : Color { return this._rgb; }
        }
\end{verbatim}

Смотри также \l class, \l function, \l{Function type}, \l{function
operator}.

\subsubsection{Квалифицированные имена}

Когда вы объявляете объект определенного типа, он сам для себя
является пространством имен. Например, в \AS есть функция
 \c Math.sin(). Если вы хотите иметь функцию sin() в вашем
 классе, это не вызовет проблем, потому что объекты вашего класса
 могут вызывать функцию с использованием синтаксиса
\c{object.function()}. Точка используется для определения пространства
имен идентификатора, следующего за ней.

Например, в GUI приложении \A , каждый объект приложения принадлежит
объекту Application. Из этого может следовать довольно длинный код,
например, \c{Application.Dialog.ListBox.count}.
Такие длинные имена часто можно сократить, например внутри
обработчика сигнала, используя \c{this.ListBox.count}.
На практике \A достаточно интеллектуально работает с неполностью
кваливицированными именами, поэтому часто код может быть сокращен
просто до \c{ListBox.count}. Вам необходимо использовать
квалифицированные имена если неквалифицированные имена неоднозначны
или когда выиспользуете импортированные через Import имена.

\subsubsection{Свойства класса}

Свойство это неопределенная переменная которая может быть
присвоена и доступна если класс поддерживает свойства.

\begin{verbatim}
var obj = new Object
object.myProperty = 100;
\end{verbatim}

В классе Object не описана переменная \c myProperty, но так как
класс поддерживает свойства мы можем описать переменную
с этим именем на лету и использовать ее позднее.
Свойства связанные с объектом принадлежат ему. Если в рассмотренном
выше примере присвоить значение свойству \c myProperty другому
экземпляру объекта, оно будет содержать другое значение не зависимо
от свойства класса obj.

\subsubsection{Константа}

Константы объявляются с использованием ключевого слова \l const:

\begin{verbatim}
    const x = "Willow";
    const y = 42;
\end{verbatim}

Значения констант должны быть описаны в месте объявления
потому что они не могут быть изменены позднее.
При попытке присвоить значение константе интерпретатор
выведет сообщение об ошибке и остановит работу.

\subsubsection{Функция}


\subsubsection{Переменная}

Переменные объявляются с помощью ключевого слова var:

\begin{verbatim}
    var a;               // Не определена
    var c = "foliage";   // Строка "foliage"
    x = 1;               // Глобальная переменная

\end{verbatim}

Если переменная имеет определенный тип, т.е. добавлен
':ИмяТипа' после ее имени, только объекты указанного типа
могут быть присвоены этой переменной.

Переменные, объявленные через var являются локальными
по отношению к окружающему их блоку кода.

\subsection{Управляющие конструкции}

Управление последовательностью выполнения программ \AS
определяется управляющими вырадениями, например,
The flow--of--control in \QS programs is controlled by control
\l if и \l switch, \l for и \l while. \AS
также поддерживает исключения с \l try и \l catch.


\subsubsection{break}

\begin{verbatim}

label:
for ( var i = 0; i < limit; i++ ) {
    if ( condition ) {
        break label;
    }
}

switch ( expression ) {
    case 1:
        Statements1;
        break;
    default:
        DefaultStatements;
	break;
}
\end{verbatim}

Это ключевое слово используется в циклах for, do, while и
операторах switch. Когда выполняется команда break в цикле,
управление передается на оператор, следующий за оператором цикла,
содержащим break; в отличие от оператора
break со следующим за ним именем метки. В этом случае
управление передается оператору, следующему за меткой.

A \c break statement is usually placed at the end of each \l{case} in
a switch statement to prevent the interpreter "falling through" to the
next case. When the interpreter encounters a \c break statement, it
passes control to the statement that follows the inner-most enclosing
\c switch statement. If every \c case has a corresponding \c break,
then at most one \c case's statements will be executed. If the \c
break statement is followed by a label name (\e label) then when the
\c break is encountered, control will pass to the statement marked
with that label; this is useful, for example, for breaking out of
deeply nested loops.

Пример:
\begin{verbatim}
    red:
    for ( x = 0; x < object.width; x++ ) {
	for ( y = 0; y < object.height; y++ ) {
	    if ( color[x][y] == 0xFF0000 ) {
		break red;
	    }
	}
    }
\end{verbatim}

Смотри switch для других примеров.
См.также do, while, for и break.


\subsubsection{case}


\subsubsection{catch}


\subsubsection{continue}


\subsubsection{default}


\subsubsection{do}


\subsubsection{else}


\subsubsection{for}


\subsubsection{if}


\subsubsection{finally}


\subsubsection{label}


\subsubsection{return}


\subsubsection{switch}


\subsubsection{throw}


\subsubsection{try}


Текст.

\subsubsection{while}

Текст.

\subsubsection{with}

Текст.


\section{Встроенные типы и объекты}


\subsection{Типы данных}


\subsubsection{Массив}


\subsubsection{Логический}


\subsubsection{Дата}


\subsubsection{Функция}


\subsubsection{Числовой}


\subsubsection{Объект}


\subsubsection{Точка}


\subsubsection{Прямоугольник}


\subsubsection{Размер}


\subsubsection{Регулярное выражение}


\subsubsection{Строка}


\subsubsection{Цвет}


\subsubsection{Палитра}


\subsubsection{Цвет группы}


\subsubsection{Массив байтов}


\subsubsection{Шрифт}


\subsubsection{Карта битов}


\subsection{Объекты}



\subsubsection{Math}


\subsubsection{System}


\section{Встроенные переменные и константы}

\AS предоставляет несколько встроенных констант и переменных.
Встроенные переменные включают \i{arguments}.
Встроенные константы включают \i{Infinity}, \i{NaN} и \i{undefined}.


\subsection{arguments}

Это массив аргументов, которые переданы функции.
Он существует только внутри контекста функции.

Пример:
\begin{verbatim}
    function sum()
    {
        total = 0;
        for ( i = 0; i < arguments.length; i++ ) {
                total += arguments[ i ];
        }
        return total;
    }
\end{verbatim}

\subsection{Application.argv}

Этот массив переменных содержит упорядоченный список аргументов
командной строки которые были переданы приложению ( если были ).
\c{Application.argv[0]} имя приложения, исполняющего скрипт;
 все остальные аргументы начинаются с \c{Application.argv[1]} и далее.

%\Bold
Пример:
\begin{verbatim}
    function main()
    {
        for ( var i = 1; i < Application.argv.length; i++ ) {
                debug( Application.argv[i] );
        }
    }
\end{verbatim}


\subsection{Константы}


\subsubsection{Infinity}


\subsubsection{NaN}


\subsubsection{undefined}


\section{Встроенные функции и операторы}


\subsection{Функции}


\subsubsection{connect()}


\subsubsection{debug()}

Отладка.

\subsubsection{eval()}


\subsubsection{isFinite()}


\subsubsection{isNaN()}


\subsubsection{killTimer()}


\subsubsection{killTimers()}


\subsubsection{parseFloat()}


\subsubsection{parseInt()}


\subsubsection{startTimer()}


\subsection{Операторы}


\subsubsection{Арифметические операторы}


\subsubsection{Операторы работы со строками}


\subsubsection{Логические операторы}


\subsubsection{Операторы сравнения}


\subsubsection{Битовые операторы}


\subsubsection{Операторы специального назначения}



\section{Доступ к объектам и функциям платформы}

Текст.

\subsection{Общая организация и пространства видимости}

Текст.

\subsection{Работа с документами}

Текст.

\subsubsection{Модуль экранной формы документа}

Модуль экранной формы документа позволяет задавать порядок обработки
таких событий как открытие, закрытие экранной формы,
выбор элементов (виджетов) экранной формы, щелчек мыши по кнопкам формы.
Для редактирования кода скрипта в Дизайнере, вызвав редактор
экранной формы документа (forms), следует перейти на закладку Module (Модуль).

\subsection{Работа с каталогами}


\subsection{Работа с регистрами}


\subsection{Работа с журналами}


\subsection{Работа с отчетами}


\subsection{Работа с метаданными}


\end{document}
